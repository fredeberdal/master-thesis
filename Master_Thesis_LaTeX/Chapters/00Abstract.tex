Code reviewing has been a prevalent method of controlling code quality, detecting bugs, and increasing the understanding of source code the last 50 years. In industrial settings, the priorities for conducting code reviews often differ from those in educational contexts. The main objective of this research is to explore the efficiency of various techniques in selecting code files for review in an educational context. In these contexts, this practice takes the form of peer reviews. To address the challenge of optimizing peer code review processes, the thesis will undertake a literature review and compare different code selection techniques. The thesis will employ a combination of structured and experimental research design. An experiment will be conducted to observe how the selection techniques will impact the review process. The selection techniques that will be applied and tested are Size Selection, Keyword Selection, Cyclomatic Complexity Selection, and Combination Selection. The techniques will be applied to three different student projects from a university course and one open source project. \\ 

The findings indicate that all the selection techniques are efficient in identifying many of the files that are crucial in a repository. However, some techniques had higher accuracy in selecting crucial files, whereas other techniques were more consistent. Although none of the techniques had perfect accuracy, the average accuracy of eight files was 5.7 for Size Selection, 5.7 for Keyword Selection, 6.3 for Combination Selection, and 6.7 for Cyclomatic Complexity Selection. The Combination Selection technique in this thesis only combined Size, Cyclomatic Complexity, and Halstead measures, which still resulted in imperfect selections. A combination of more techniques might have worked even better, but there are many considerations to be made before one should apply these in practice. 

%An optimal selection list is created for each project, and comparisons are made between the files selected by each technique and the optimal list. The consistency of each selection technique is also measured by noting the frequency with which they identify the same files.

%Write an abstract/summary of your thesis, and state your main findings here. \\

%A summary should be included in both English and any second language, if this is applicable, regardless if the thesis is written in English or in your preferred language. These should be on separate pages, the English version first.