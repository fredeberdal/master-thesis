Kodegjennomgang har vært en utbredt metode for å kontrollere kodekvalitet, oppdage feil og øke forståelsen av kildekode de siste 50 årene. I industrielle settinger er prioriteringene for å gjennomføre kodegjennomganger ofte forskjellige fra de i utdanningskontekster. Hovedmålet med denne avhandlingen er å utforske effektiviteten av ulike teknikker for å velge ut kodefiler for gjennomgang i en utdanningskontekst. I disse kontekstene tar denne praksisen form av studentkodevurdering. For å møte utfordringen med å optimalisere prosessene for studentkodevurdering av kode, vil avhandlingen gjennomføre en litteraturgjennomgang og sammenligne forskjellige kodeseleksjonsteknikker. Avhandlingen vil benytte en kombinasjon av strukturert og eksperimentelt forskningsdesign. Et eksperiment vil bli gjennomført for å observere hvordan seleksjonsteknikkene påvirker gjennomgangsprosessen. Seleksjonsteknikkene som anvendes og testes er Størrelsesbasert seleksjon, Nøkkelordbasert seleksjon, Syklomatisk kompleksitetsseleksjon og Kombinasjonsseleksjon. Teknikkene anvendes på tre forskjellige studentprosjekter fra et universitetskurs og ett åpen kildekodeprosjekt. \\

Resultatene indikerer at alle seleksjonsteknikkene er effektive til å identifisere mange av de filene som er avgjørende i et prosjekt. Noen teknikker hadde imidlertid høyere nøyaktighet i å velge avgjørende filer, mens andre teknikker var mer konsistente. Selv om ingen av teknikkene hadde perfekt nøyaktighet, var gjennomsnittlig nøyaktighet av åtte filer 5.7 for Størrelsesbasert seleksjon, 5.7 for Nøkkelordbasert seleksjon, 6.3 for Kombinasjonsseleksjon og 6.7 for Syklomatisk kompleksitetsseleksjon. Kombinasjonsteknikken i denne avhandlingen kombinerte bare Størrelse, Syklomatisk kompleksitet og Halstead, noe som allikevel ikke resulterte i et perfekt utvalg. En kombinasjon av flere teknikker kunne ha fungert enda bedre, men det er mange hensyn å betrakte før man kan anvende disse i praksis.